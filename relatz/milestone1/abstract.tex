Com o eventual aparecimento dos computadores quânticos, os esquemas de assinaturas baseados no problema da factorização ou do logaritmo discreto tornar-se-ão inúteis. Posto isto, tornou-se imperativo procurar esquemas de assinaturas digitais que se apresentassem como alternativas viáveis. Esta procura levou a que se chegassem a esquemas baseados em reticulados, que se revelaram uma alternativa válida ao problema que se pretendia resolver e que proporcionaram avanços significativos em várias áreas da criptografia. \\
Este Projecto Integrador pretende abordar e explorar a aplicabilidade de esquemas criptográficos de autenticação de mensagens baseados em reticulados. Neste relatório será apresentado todo o trabalho de análise e estudo efectuado sobre as duas publicações sugeridas \cite{lapin,lattice_sig}.\\
Do estudo efectuado para esta primeira \textit{milestone}, resultou um pequeno protótipo em \sage\ do protocolo Lapin descrito em \cite{lapin}. Além disso, são também apresentados os objectivos que se esperam alcançar no desenvolvimento deste projecto.
