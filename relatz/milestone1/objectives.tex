\section{Objectivos}
Nesta primeira fase pretende-se estudar o protocolo de autenticação Lapin. Para isso, vamos implementar um pequeno protótipo em \sage, visto que já utilizámos \sage\ para o primeiro Projecto Integrador. Depois de implementado o protótipo, é necessário escolher uma linguagem para a implementação do protocolo.\\
As opções consideradas para a implementação do protocolo eram \textsf{C}, \textsf{C++} e \textsf{Java}. Estamos mais confortáveis em \textsf{Java}, mas visto que o Lapin é especialmente orientado para dispositivos \textsf{low-cost}, decidimos usar \textsf{C}.\\
Para esta primeira fase também efectuámos um estudo das operações aritmétricas mais eficientes, usando apenas operações binárias. Estas operações até poderão ser inicialmente criadas em \sage, com o objectivo de conhecer e perceber melhor os algoritmos das mesmas.\\
No resto do presente documento, apresenta-se todo o estudo e trabalho desenvolvido para esta primeira fase. 