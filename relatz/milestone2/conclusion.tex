%%%%%%%%%%%
% Conclusao
%%%%%%%%%%%
Baseado no estudo efectuado sobre o protocolo de autenticação Lapin, pode-se afirmar que o mesmo é uma opção viável e segura para implementar em sistemas \textit{low-cost} e com recursos limitados, em que se pretenda um protocolo com pequena complexidade comunicacional. A análise de \cite{lapin} permitiu também verificar a utilidade de algumas operações realizadas sobre polinómios que permitem alcançar resultados muito mais eficientes comparativamente às operações realizadas da forma habitual. De assinalar ainda que a utilização da suposição \textsf{Ring-LPN}, apesar de ainda não ser um problema muito utilizado em criptografia, vem demonstrar que a sua utilização pode vir a aumentar, especialmente no âmbito das construções de esquemas \textit{low-cost}.\\
Relativamente ao esquema de assinaturas baseadas em \textit{Lattices} foi possível concluir que este é solução bastante interessante e com bastante potencial, contudo as limitações para os casos que conseguimos de facto testar apresentaram resultados que não são ainda aceitáveis para uma utilização prática consistente e eficiente. De ressalvar também que a nossa análise não é extensível a todos os casos referidos na publicação, sendo que os restantes são baseados em problemas diferentes dos testados, e por isso os resultados poderão ser diferentes. Claramente esta é uma área de investigação em evolução e com grande potencial.
%%%%%%%%%%%%%%%%%%%%
% Implementação
%%%%%%%%%%%%%%%%%%%%
\section{Resultados de implementação e trabalho futuro}
A implementação do Lapin acabou por não se tornar muito complexa, sendo bastante simples perceber o funcionamento do protocolo olhando apenas para o código. No entanto as operações aritmétricas sobre polinómios acabaram por ser a parte mais complicada de implementar, sendo que a função de redução modular \verb|poly_mod| pode ser melhorada. Outra opção que com certeza irá aumentar a eficiência do protocolo será a utilização dos algoritmos de redução modular mais rápidos (\textit{fast modulo reduction} \cite{Hankerson:2003:GEC:940321}). Resta-nos terminar a implementação do algoritmo de Euclides estendido, do CRT e a utilização do gerador \textit{urandom} para geração aleatória de polinómios. Além disso, há operações que podem sofrer pequenas optimizações. Entendemos ainda que seria relevante modificar a nossa implementação de forma a que se tornasse em algo menos académico e mais direccionado à utilização prática no dia-a-dia em dispositivos mais restringidos.\\
Em relação às assinaturas digitais baseadas em reticulados seria extremamente interessante aprofundar o estudo relativamente às distribuições, o que influenciaria directamente a geração do vector \textbf{y} e também a \textit{rejection sampling}. Mais precisamente, seria interessante implementar o algoritmo de amostragem gaussiana apresentado em \cite{galbraith2012efficient}. Esta questão das distribuições revelou-se algo complexa e não foi possível abordá-las mais profundamente, mas acreditamos que o protocolo se revelaria mais eficiente com essas modificações.\\ 
Gostaríamos também de conseguir aprofundar o nosso conhecimento sobre a teoria dos reticulados, dado que à partida para este projecto o nosso conhecimento sobre a matéria era praticamente zero. De qualquer forma, achamos que os resultados obtidos foram bastante mais do que satisfatórios.