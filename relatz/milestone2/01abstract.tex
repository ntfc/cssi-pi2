\begin{abstract}
\thispagestyle{plain} % pagenumber on abstract page
Com o eventual aparecimento dos computadores quânticos, os esquemas de assinaturas baseados no problema da factorização ou do logaritmo discreto tornar-se-ão inúteis. Dado que a maioria dos esquemas baseiam-se na teoria de número, acredita-se que com esse eventual aparecimento estes problemas deixarão de ser difíceis e por isso sem utilidade na criptografia. Posto isto, tornou-se imperativo procurar esquemas de assinaturas digitais que se apresentassem como alternativas viáveis. Esta procura levou a que se chegassem a esquemas baseados em reticulados, que se revelaram uma alternativa válida ao problema que se pretendia resolver e que proporcionaram avanços significativos em várias áreas da criptografia.\\

Este Projecto Integrador pretende abordar e explorar a aplicabilidade de esquemas criptográficos de autenticação de mensagens baseados em reticulados. Assim, neste relatório será apresentado todo o trabalho de análise e estudo efectuado sobre as duas publicações sugeridas \cite{lapin,lattice_sig}. Do estudo resultaram dois protótipos em \sage: um do protocolo Lapin \cite{lapin} e outro do protocolo de assinatura baseado em reticulados \cite{lattice_sig}. Além disso será apresentada a implementação em \textsf{C} do protocolo Lapin.
\end{abstract}
%
%
%
%
%%% Antigo