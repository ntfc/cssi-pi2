\chapter{Introdução}
\section{Objectivos}
Numa primeira fase o objectivo pretendido consistiu em estudar um novo protocolo autenticação de forma a poder desenvolver um pequeno protótipo  em \sage, uma ferramenta com a qual estávamos familiarizados dado que foi utilizada no primeiro Projecto Integrador. Nesta fase foi também possível efectuar um estudo das operações aritméticas sobre polinómios mais eficientes, apenas com o uso de operações binárias.\\
Neste última fase era pretendido que fosse feita a implementação do protocolo estudado na primeira fase num linguagem mais eficiente. Foram consideradas para a implementação do protocolo as linguagens \textsf{C}, \textsf{C++} e \textsf{Java}. Apesar de a linguagem em que nos sentimos mais confortáveis ser \textsf{Java}, considerámos que esta é a menos eficiente e a menos indicada para dispositivos \textit{low-cost}, portanto optámos por \textsf{C} (nunca utilizámos \textsf{C++} e não nos pareceu que oferecesse vantagens em relação ao \textsf{C}).\\
O resto do presente documento apresenta todo o estudo efectuado sobre as publicações, os protótipos desenvolvidos e ainda a implementação do protocolo de autenticação Lapin.\\
%%%%%%%%%%%%%%%%%%%%%%%%%%%%%%%%%%%%%5
\section{Computação quântica}
A possibilidade do aparecimento de computadores quânticos é um tema sensível e que gera muitas expectativas no seio da comunidade criptográfica. Isto deve-se ao facto de se acreditar que esta possibilidade deverá resolver determinados problemas usados em criptografia que se julgam difíceis, o que tornará vários esquemas inseguros.\\
Vários esquemas são baseados na dificuldade de problemas como a factorização de inteiros ou logaritmo discreto. A existência de computadores quânticos resolveria de forma eficiente estes problemas e assim inviabilizaria a utilização de algoritmos como o \textsf{RSA}. Isto afectaria vários sistemas cuja segurança é garantida pelo seu uso. Posto isto surgiu a necessidade de se encontrar soluções alternativas que não seja afectadas pela existência da computação quântica. Esta nova linha de investigação, cuja incidência sobre esquemas baseados em reticulados é bastante acentuado, tem conseguido resultados extremamente positivos, contudo é ainda prematuro garantir a sua aplicabilidade.
\section{Protocolo de autenticação}
Um protocolo de autenticação tem como objectivo a autenticação de duas entidades que pretendem efectuar uma comunicação segura. São protocolos de chave privada em que um \textit{prover} $\mathcal{P}$ autentica-se perante um \textit{verifier} $\mathcal{V}$.\\
Uma das famílias de protocolos de autenticação existente é o \textit{Challenge-Response}, que consiste num protocolo em que uma parte gera um \textit{challenge} que envia à outra parte, sendo que esta envia uma resposta que caso seja válida a autentica perante a primeira. Um exemplo trivial desta família de protocolos é o processo de autenticação de password, em que o challenge é o pedido de inserção da password correcta, sendo que caso esta seja válida a autenticação é alcançada.\\

\section{Protocolo de assinaturas}
Protocolos de assinaturas pretendem garantir que a autenticidade de uma mensagem. Através do uso destes protocolos o receptor tem a garantia de que quem enviou a mensagem é alguém por ele conhecido, este não pode negar esse envio, e ainda que a mensagem não foi alterada durante a comunicação. Ou seja, garante autenticação, não-repúdio e integridade da mensagem, respectivamente.\\
Os esquemas de assinaturas consistem tipicamente em três algoritmos: geração de chaves que gera o par de chaves, o algoritmo de assinatura que recebe a chave privada e a mensagem e gera a assinatura, e o algoritmo de verificação que recebe a mensagem, a assinatura e a chave pública e verifica se a assinatura é valida ou não.\\

No âmbito dos protocolos baseados em reticulados, já existem alguns esquemas criptográficos seguros e com aplicação prática, mas poucos esquemas de assinaturas seguros e com aplicação prática.